\documentclass[14pt, a4paper]{extarticle}

\input{../preamble_main}

\begin{document}
	\onehalfspacing
	\include{title_page}
	\setcounter{page}{2}
	
	\section*{Задание}
	
	Дан объект:
	$$\begin{matrix}
		\dot{x} = Ax + bu, x(0),\\
		y = Cx,
	\end{matrix}$$
	где $x\in R^n$ -- вектор состояния, $u$ -- управление, $y \in R$ -- регулируемая переменная,
	$$A=\left[\begin{matrix}
		0 & 1 & 0 & \cdots & 0 \\
		0 & 0 & 1 & \cdots & 0 \\
		\vdots & \vdots & \vdots & \ddots & \vdots \\
		0 & 0 & 0 & \cdots & 1 \\
		-a_0 & -a_1 & -a_2 & \cdots & -a_{n-1} \\
	\end{matrix}\right], b = \left[\begin{matrix}
		0 \\ 0 \\ \vdots \\ 0 \\ b_0 \\
	\end{matrix}\right], C = \left[\begin{matrix}
		1 & 0 & \cdots & 0 & 0
	\end{matrix}\right],$$
	$a_i, i=\overline{0,n-1}$ -- неизвестные параметры, $b_0$ -- известный коэффициент.
	
	Задача управления заключается в компенсации параметрической неопределенности объекта и обеспечении следующего целевого равенства:
	$$\lim\limits_{t\to\infty}\left|\left|x_M(t)-x(t)\right|\right|=\lim\limits_{t\to\infty}\left|\left|e(t)\right|\right|=0,$$
	где $e=x_M-x$ -- вектор ошибки управления, $x_M\in R^n$ --вектор, генерируемый эталонной моделью:
	$$\begin{matrix}
		\dot{x}_M = A_Mx_M + b_Mg,\\
		y_M = C_Mx_M
	\end{matrix}$$
	с задающим воздействием $g(t)$ и матрицами:
	$$A_M=\left[\begin{matrix}
		0 & 1 & 0 & \cdots & 0 \\
		0 & 0 & 1 & \cdots & 0 \\
		\vdots & \vdots & \vdots & \ddots & \vdots \\
		0 & 0 & 0 & \cdots & 1 \\
		-a_{M0} & -a_{M1} & -a_{M2} & \cdots & -a_{Mn-1} \\
	\end{matrix}\right], b_M = \left[\begin{matrix}
		0 \\ 0 \\ \vdots \\ 0 \\ a_{M0} \\
	\end{matrix}\right],$$
	$$C_M = \left[\begin{matrix}
		1 & 0 & \cdots & 0 & 0
	\end{matrix}\right]$$

	Параметры эталонной модели $a_{Mi},i=\overline{1,n-1}$ строятся на основе метода стандартных характеристических полиномов для обеспечения желаемого качества воспроизведения задающего воздействия $g(t)$. Другими словами, модель определяет желаемое качество замкнутой системы после завершения процессов настройки адаптивного управления.

	\begin{enumerate}
		\item На основе заданных в таблице значений времени переходного процесса $t_\text{п}$ и максимального перерегулирования $\bar{\sigma}$ сформировать эталонную модель. Построить график переходной функции модели, на котором показать время переходного процесса $t_\text{п}$ и перерегулирование $\bar{\sigma}$;
		\item На основе предположения, что параметры объекта известны, построить и промоделировать систему управления с регулятором. Провести три эксперимента, в которых:
		\subitem -- использовать расчетные значения параметров объекта, заложенные в $\theta_1$ и $\theta_2$;
		\subitem -- незначительно отклонить параметры объекта так, чтобы система не потеряла устойчивость;
		\subitem -- отклонить параметры объекта так, чтобы система потеряла устойчивость.
		
		По результатам каждого эксперимента построить траектории $x(t)$ и $x_M(t)$ на одном графике и $e(t)$ -- на другом;
		\item Провести моделирование адаптивной системы управления с регулятором и алгоритмом адаптации. В ходе моделирования проиллюстрировать свойства 1-4 алгоритма управления. Для этого необходимо:
		\subitem -- повторить три эксперимента п.п.~2 для фиксированного значения $\gamma$;
		\subitem -- используя расчетные значения параметров объекта, провести эксперимент с тремя различными значениями $\gamma$;
		\subitem -- провести один из предыдущих экспериментов данного пункта при $g(t)=1$.
		
		По результатам каждого эксперимента построить траектории $x(t)$ и $x_M(t)$ на одном графике, $e(t)$ -- на втором, $\tilde{\theta}=\theta-\hat{\theta}$ -- на третьем;
		\item Сделать выводы по каждому пункту работы.
 	\end{enumerate}
	\begin{table}[H]
		\centering
		\begin{tabular}{|l|l|p{0.1\textwidth}|p{0.13\textwidth}|p{0.2\textwidth}|p{0.15\textwidth}|}
			\hline
			Вариант & Матрица $A$ & Коэф передачи $b_0$ & Время переходного процесса, $t_n$ & Максимального перерегулирование $\bar{\sigma}$, \% & Сигнал задания $g(t)$\\\hline
			18 & 
			$\left[
			\begin{matrix}
				0 & 1 \\
				-10 & 6 
			\end{matrix}
			\right]$
			& 9 & 0,15 & 15 & $0,8sin2t+cos0,8t+2$ \\\hline
		\end{tabular}
	\end{table}
	
	\newpage
	
	\section*{Описание работы}
	
	
	
	\newpage
	
	\section*{Вывод}
	

\end{document}